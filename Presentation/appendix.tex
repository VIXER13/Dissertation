\section{Вопросы и замечания}
\begin{frame}{Вопросы и замечания}
	\begin{itemize}
		\justifying
		\item Во введении работы фигурирует термин <<структурно-чувствительные материалы>>. Было бы уместно более чётко определить это понятие, так как не до конца ясно, какой класс материалов следует называть структурно-чувствительным. (\textbf{Ведущая организация})
		\item Основное назначение предложенных автором моделей -- это моделирование процессов в микро- и нано-неоднородных средах и материалах. Вместе с тем, связь между параметрами использованных феноменологических моделей и параметрами первичными, «микронеоднородных» моделей в работе не показана и не анализируется. (\textbf{Савенков\,Е.\,Б.})
		\item Автореферат представляется несколько перегруженным в плане различных формул, возможно, в ущерб объяснениям полученных результатов с точки зрения физического смысла. (\textbf{Ступин\,Д.\,Д.})
		\item В автореферате нет указаний на примеры решения каких-либо практических задач. (\textbf{Стрижак\,С.\,В.})
	\end{itemize}
\end{frame}

\begin{frame}
	\begin{itemize}
		\justifying
		\item В определении интегрального нелокального оператора (1.1) фигурирют следующие параметры: весовой параметр $p_1$ и $p_2$, функция нелокального влияния $\varphi$ и область нелокального влияния $S'(\boldsymbol{x})$, которые в дальнейшем становятся частью уравнений теплопроводности и равновесия. Какие из этих параметров являются материальными и могут быть установлены из экспериментов? (\textbf{Ведущая организация})
		\item В работе был проведён анализ с исследованием поведения решений при использовании двух семейств функций нелокального влияния. Однако неясно, из каких соображений следует выбирать то или иное. (\textbf{Бураго\,Н.\,Г.})
		\item В автореферате были приведены два семейства функций нелокального влияния, но не были описаны различия между ними. В связи с этим представляется неясным, из каких соображений нужно выбирать ту или иную функцию нелокального влияния и другие параметры нелокальной модели. (\textbf{Ступин\,Д.\,Д.})
		\item Не вполне понятно, их каких физических экспериментов можно установить параметры, характеризующие нелокальные свойства материалов (вид функции нелокального влияния, характерный размер носителя этой функции, весовые параметры). (\textbf{Стрижак\,С.\,В.})
	\end{itemize}
\end{frame}

\begin{frame}
	\begin{itemize}
		\justifying
		\item Следовало подробное объяснение происхождение формулы (1.1), так как как выясняется далее, оно является фундаментальным соотношением для данной диссертационной работы. \\ (\textbf{Ведущая организация})
		\item Следует также отметить, что без каких-либо объяснений представлены формулы (1.3), (1.5) и (1.6). Ссылки на замечательные работы [37-39] не совсем уместны, так как в этих работах рассматриваются <<определяющие уравнения>>, <<уравнение теплопроводности>> и <<уравнения движения>> для математической модели нелокальной термовязкоупругой среды. (\textbf{Ведущая организация})
	\end{itemize}
\end{frame}

\begin{frame}
	\begin{itemize}
		\justifying
        \item Из содержания текста перед формулой (1.2) видно, что эту формулу следовало записать для двумерной области, однако содержание текста после этой формулы указывает на то, что оно представлено для трёхмерной области. (\textbf{Ведущая организация})
        \item Не указаны в формулах индексы какие значения принимают, а также не указано на суммирование по повторяющимися индексам. \\ (\textbf{Ведущая организация})
        \item На 27-ой странице при получении последних двух соотношений, которые применяются в дальнейшем, интегрирование по частям некорректно осуществлено. (\textbf{Ведущая организация})
        \item На 31-ой странице на 7-ой и 8-ой строках сверху написано: <<... представим векторы ... в виде тензоров второго ранга, ...>> Это высказывание некорректно, так как векторы нельзя представить в виде тензоров второго ранга. (\textbf{Ведущая организация})
        \item Предложенный в работе алгоритм численного решения оперирует блочными матрицами и в работе были введены определения блоков, из которых ассемблируются матрицы теплопроводности (2.9) и жёсткости (2.10). Однако процедура, при которой были получены именно такие определения блоков, не до конца изложена. (\textbf{Савенков\,Е.\,Б.})
	\end{itemize}
\end{frame}

\begin{frame}
	\begin{itemize}
		\justifying
		\item С учётом того, что основой численного метода является метод конечных элементов, не совсем ясна целесообразность создания собственного программного комплекса вместо написания модуля к уже существующим. Это бы позволило решать более сложные задачи и сосредоточить внимание на математической стороне вопроса. Сам выбор метода конечных элементов не совсем понятен: почему приоритет отдан именно ему, а не, например, методу контрольных объёмов, обеспечивающему строгое выполнение балансовых соотношений. \\ (\textbf{Стрижак\,С.\,В.})
	\end{itemize}
\end{frame}

\begin{frame}
	\begin{itemize}
		\justifying
		\item Не ясна причина использования именно квадратичных серендиповых элементов. Например, если проводить расчёты билинейными элементами, будет ли большая разница между решениями? Или, если использование квадратичных элементов необходимо, то почему использованы восьмиузловые серендиповы, а не девятиузловые лагранжевы элементы? (\textbf{Бураго\,Н.\,Г.})
		\item В автореферате не указаны допустимые соотношения размеров элементов сетки и радиуса нелокального влияния. Также не совсем ясен принцип, по которому можно перейти от одного способа аппроксимации области нелокального влияния к другому. (\textbf{Федотенков\,Г.\,В.})
	\end{itemize}
\end{frame}

\begin{frame}
	\begin{itemize}
		\justifying
		\item Не ясным остаётся способ оценки эффективности распараллеливания алгоритма ассемблирования матриц теплопроводности и жёсткости. Почему в локальном случае рассмотрения классических матриц эффективность распараллеливания ниже, чем в нелокальном? \\ (\textbf{Федотенков\,Г.\,В.})
		\item Отсутствуют какие-либо оценки абсолютных затрат времени на выполнение расчётов, хотя бы по сравнению с решением аналогичных по постановке задач, но в рамках классических (<<локальных>>) моделей. Нет данных по эффективности распараллеливания алгоритма при использовании технологии MPI. (\textbf{Стрижак\,С.\,В.})
	\end{itemize}
\end{frame}

\begin{frame}
    \begin{itemize}
    \justifying
        \item Для предобуславливания системы линейных алгебраических уравнений конечно-элементных аппроксимаций использован алгоритм неполного разложения Холецкого. Детали его реализации не приводятся. Вместе с тем, в настоящее время существуют достаточно эффективные параллельные (MPI, OpenMP) реализации неполного разложения Холецкого, например, в свободной и бесплатной библиотеке SuperLU. Использование подобных библиотек сделало бы параллельной самую вычислительно «тяжелую» часть программной реализации и позволило бы рассматривать задачи существенно большей сеточной размерности. Так же автору следует рассмотреть возможность использования предобуславливателей на основе многосеточного метода, имеющих практически идеальную масштабируемость и «по-элементных» («element-by-element») предобуславливателей. \\ (\textbf{Савенков\,Е.\,Б.})
    \end{itemize}
\end{frame}
