\begin{frame}
    \frametitle{Ответы на замечания ведущей организации}
    \begin{itemize}
        \item Замечание 1
        \item Замечание 2
        \item Замечание 3
        \item Замечание 4
        \item Замечание 5
    \end{itemize}
\end{frame}

\begin{frame}
    \frametitle{Ответы на замечания оппонента Бураго\,Н.\,Г.}
    \begin{itemize}
    	\justifying
        \item Не ясна причина использования именно квадратичных серендиповых элементов. Например, если проводить расчёты билинейными элементами, будет ли большая разница между решениями? Или, если использование квадратичных элементов необходимо, то почему использованы восьмиузловые серендиповы, а не девятиузловые лагранжевы элементы?
        \item В работе был проведён анализ с исследованием поведения решений при использовании двух семейств функций нелокального влияния. Однако неясно, из каких соображений следует выбирать то или иное.
    \end{itemize}
\end{frame}

\begin{frame}{Ответы на замечания оппонента Савенкова\,Е.\,Б.}
    \begin{itemize}
    \justifying
        \item Для предобуславливания системы линейных алгебраических уравнений конечно-элементных аппроксимаций использован алгоритм неполного разложения Холецкого. Детали его реализации не приводятся. Вместе с тем, в настоящее время существуют достаточно эффективные параллельные (MPI, OpenMP) реализации неполного разложения Холецкого, например, в свободной и бесплатной библиотеке SuperLU. Использование подобных библиотек сделало бы параллельной самую вычислительно «тяжелую» часть программной реализации и позволило бы рассматривать задачи существенно большей сеточной размерности. Так же автору следует рассмотреть возможность использования предобуславливателей на основе многосеточного метода, имеющих практически идеальную масштабируемость и «по-элементных» («element-by-element») предобуславливателей.
    \end{itemize}
\end{frame}

\begin{frame}{Ответы на замечания оппонента Савенкова\,Е.\,Б.}
	\begin{itemize}
		\item Основное назначение предложенных автором моделей -- это моделирование процессов в микро- и нано-неоднородных средах и материалах. Вместе с тем, связь между параметрами использованных феноменологических моделей и параметрами первичными, «мкиронеоднородных» моделей в работе не показана и не анализируется.
		\item Предложенный в работе алгоритм численного решения оперирует блочными матрицами и в работе были введены определения блоков, из которых ассемблируются матрицы теплопроводности (2.9) и жёсткости (2.10). Однако процедура, при которой были получены именно такие определения блоков, не до конца изложена.
	\end{itemize}
\end{frame}
