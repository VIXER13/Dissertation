\begin{frame}[noframenumbering,plain]
    \setcounter{framenumber}{1}
    \maketitle
\end{frame}

\begin{frame}
    \frametitle{Положения, выносимые на защиту}
    \begin{itemize}
    	\justifying
        \item Модели нелокальной теплопроводности и термоупругости, позволяющие описать процессы передачи теплоты и напряжённо-деформированного состояния в структурно-чувствительных материалах.
        \item Новые численные алгоритмы решения на основе метода конечных элементов, адапатированные под многопроцессорные вычислительные системы.
        \item Собственный программный комплекс NonLocFEM, в рамках которого реализованы все рассматриваемые в работе методы решений.
    \end{itemize}
\end{frame}
\note{
    Проговариваются вслух положения, выносимые на защиту
}

\begin{frame}{Содержание}
	\LARGE
	\begin{enumerate}
		\item Основные соотношения
		\item Численный алгоритм решения
		\item Программный комплекс
		\item Анализ решений
		\item Заключение
	\end{enumerate}
    %\tableofcontents
\end{frame}
%\note{
%    Работа состоит из четырёх глав.
%
%    \medskip
%    В первой главе \dots
%
%    Во второй главе \dots
%
%    Третья глава посвящена \dots
%
%    В четвёртой главе \dots
%}
