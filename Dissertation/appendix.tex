\chapter{}\label{app:A}

Ниже представлен Листинг 1.1 конфигурационного файла, на основе которого был произведён расчёт комбинированной задачи темплопроводности и термоупругости из раздела \ref{sec:ResultsAnalysis/ThermalKirshProblem}. Конфигурационный файл выполнен в виде структуры, описанной в формате JSONSchema, и содержит 6 основных полей: task, save, mesh, thermal\_boundaries, mechanical\_boundaries и materials.

В поле task описаны основные характеристики запускаемой задачи: её размерность, тип расчёта и зависимость расчёта от времени. В поле save содержатся параметры сохранения результатов расчётов, указан путь сохранения результатов, а также названия сохраняемых файлов и точность с которой записывать результаты расчётов. В разделе mesh указан путь по которому находится файл содержащий конечно-элементную сетку. Поля thermal\_boundaries и mechanical\_boundaries содержат граничные условия для температурной и механической задач соответственно. Здесь важно отметить, что граничные условия заданы на именованных границах, поэтому важно, чтобы сетка содержала информацию об этих границах в виде групп элементов. В разделе materials указаны физические параметры материала и модельные параметры отдельно для уравнения теплопроводности и уравнения равновесия. Здесь также важно отметить, что информация о материале задана на именованной группе элементов, которые обозначают определённый материал. Таких групп может быть несколько, где для каждого материала могут быть заданы свои параметры.

\begingroup
\captiondelim{ } % разделитель идентификатора с номером от наименования
\lstinputlisting[lastline=87,language={},caption={Конфигурационный файл для комбинированной задачи теплопроводности и термоупругости},label={lst:external1}]{listings/thermomechanical_2d.json}
\endgroup
