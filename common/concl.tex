%% Согласно ГОСТ Р 7.0.11-2011:
%% 5.3.3 В заключении диссертации излагают итоги выполненного исследования, рекомендации, перспективы дальнейшей разработки темы.
%% 9.2.3 В заключении автореферата диссертации излагают итоги данного исследования, рекомендации и перспективы дальнейшей разработки темы.
\begin{enumerate}
	\item Рассмотрена иерархия моделей нелокальной теплопроводности и термоупругости, предложено и проанализировано два семейства возможных функций нелокального влияния, заданных на областях ограниченных суперэллипсами.
	
	\item Разработан численный алгоритм решения интегро-дифференциальных уравнений на основе метода конечных элементов, проведена работа над его оптимизацией и подготовкой к использованию в параллельной среде вычислений.
	
	\item Разработан программный комплекс NonLocFEM, в рамках которого реализованы все предложенные алгоритмы, параллельные реализации алгоритмов задействуют технологии параллельного программирования OpenMP и MPI, все исследования и расчёты проведены в рамках программного комплекса.
	
	\item Проведён качественный анализ сравнения классических теорий теплопроводности и термоупругости с их нелокальными постановками, полученные результаты свидетельствуют о снижении роли концентраторов в распределениях полей напряжений и плотности теплового потока; возникновении кромочных эффектов на свободных от граничных условий границах, а также определены основные зависимости отклонений нелокальных решений относительно классическим путём вариации параметров модели.
	
	\item Исследован вопрос сходимости итерационных методов решения СЛАУ применительно к задачам в нелокальных постановках, предложены способы ускорения сходимости с применением альтернативных базисов конечных элементов и предобуславливателей.
\end{enumerate}
