
{\actuality}
\ifsynopsis
Задачи термоупругости возникают в самых разных областях инженерного дела: аэрокосмической отрасли, строительстве, микроэлектронике и многих других. Передовые технологии таких отраслей, как правило, тесно связаны с созданием новых материалов. При этом часто требования уже настолько высоки, что при их создании необходимо учитывать молекулярную структуру материала, напрямую влияющую на свойства среды. Такие материалы принято называть структурно-чувствительными.
\else
Задачи термоупругости очень популярны в различных инженерных приложениях, так как температурные деформации могут существенным образом повлиять на функциональные свойста рассматриваемых объектов, вплоть до их полного выхода из строя. Особенно популярны такого рода задачи в аэрокосмической отрасли при моделировании поведения обшивок корпусов и двигателей летательных аппаратов \cite{Aerocosmos1, Aerocosmos2, Aerocosmos3}, так как они подвержены очень высоким и в то же время неравномерным нагружениям \cite{Aerocosmos4, Aerocosmos5, Aerocosmos6, Aerocosmos7}. Помимо аэрокосмической отрасли такие задачи могут быть востребованы в строительстве, особенно в строительстве критической инфраструктуры, такой, например, как атомные электространции \cite{StroyMech1, StroyMech2}. В гражданской инфраструктуре рассматривают анализ влияния таких разрушительных явлений как пожары \cite{StroyMech3, StroyMech4} или обычная циклическая смена сезона \cite{StroyMech5, StroyMech6} при воздействии которых конструкция не должна потерять устойчивость. В микроэлектронике эти задачи также не остаются без внимания \cite{MicroElectronic1, MicroElectronic2}, а учитывая возрастающее количество вычислительных мощностей и популярность носимой электроники, возникают задачи об эффективном отводе тепловой энергии \cite{MicroElectronic3}.
\fi

\ifsynopsis
Важным этапом в создании новых материалов является построение математической модели, способной адекватно описывать их поведение. Классические материалы можно описать моделями механики сплошной среды, однако, когда речь идет о структурно-чувствительных материалах, где величина структуры не превышает нескольких десятков нанометров, гипотеза сплошности нарушается, из-за чего приходится прибегать к другим моделям, например, моделям молекулярной динамики или статистическим моделям. Анализ такого рода моделей очень ограничен без численного эксперимента, а для проведения полноценного эксперимента необходимы большие вычислительные мощности, которые не всегда доступны исследователю. В связи с этим в середине XX века набирают популярность модели обобщённой механики сплошной среды, которые могут учесть такие явления, как микровращения отдельных зёрен материала, микродислокации, различные дальнодействующие и многие другие масштабные эффекты.
\else
Все перечисленные ранее и многие другие задачи объединяет потребность в создании новых материалов, которые будут отвечать соответствующим их использованию требованиям. На сегодняшний день в некоторых отраслях требования к свойствам материалов становятся уже настолько высокими, что при их создании необходимо учитывать молекулярную структуру материала на микро- и наноуровне \cite{MaterialStructure1, MaterialStructure2, MaterialStructure3}, так как их свойства могут напрямую зависеть от этого. Такие материалы принято называть структурно-чувствительными, а создание материалов с наперёд заданными свойствами на сегодняшний день является одной из сложнейших, но вместе с этим крайне актуальной областью материаловедения \cite{Auxetics}.
\fi

\ifsynopsis
В диссертационной работе рассмотрен класс моделей, обеспечивающих описание дальнодействующих эффектов путём обобщения классических уравнений механики сплошной среды и представлении их в интегро-дифференциальной форме. Такие модели принято называть нелокальными, а их разработка активно велась в рамках работ следующего списка авторов: E.~Kr{\"o}ner, D.G.B.~Edelen, A.C.~Eringen, D.~Rogula, S.B.~Altan, C.~Polizzotto, A.A.~Pisano, В.В.~Васильев, С.А.~Лурье, С.Л.~Соболев, В.С.~Зарубин, Г.Н.~Кувыркин, И.Ю.~Савельева и многие другие.
\else
Вместе с проблемой создания структурно-чувствительных материалов многие исследователи сталкиваются с проблемой моделирования их поведения. При рассмотрении наномасштабных структур отсутствует возможность использовать гипотезу сплошности среды, из-за чего классические модели механики сплошной среды не могут даже на качественном уровне передать все особенности их поведения. Так, например, в наномасштабе перестаёт работать гипотеза Био --- Фурье \cite{FourierLaw1, FourierLaw2} и наблюдается пониженная чувствительность к концентраторам напряжений \cite{ConcentrationInsensitive1, ConcentrationInsensitive2}. Этому способствуют такие эффекты, как микровращения отдельных зёрен материала, микродислокации, различные дальнодействующие и многие другие масштабные эффекты, которые могут быть смоделированы только при помощи новых математических моделей.
\fi

\ifsynopsis
В практических приложениях на основе математической модели необходимо решать большую серию задач, не все из них обладают аналитическими решениями. В связи с этим необходимо развивать подходы с использованием численных методов решения уравнений с последующей реализацией в виде программного комплекса. В диссертационной работе этому аспекту уделено особое внимание. За основу численной схемы был взят метод конечных элементов, его реализация стала частью программного комплекса NonLocFEM.
\nocite{AMCSM2019}
		\nocite{ZAMM}
		\nocite{NonlocalSaintVenant}
		\nocite{NonlocalRadiation}
\else
На сегодняшний день существует большое количество моделей способных описать различные масштабные эффекты. Однако подходы моделирования могут достаточно сильно различаться между собой при рассмотрении разных линейных размеров и временных отрезков. Таким образом, возникают иерархии моделей, способных качественно и количественно описать различные аспекты поведения материала на разных масштабах. Это, в свою очередь, приводит к идее многомасштабного моделирования \cite{Multiscale1}, где, например, некоторые характеристики материала можно вычислять при помощи моделей, находящихся ниже по иерархии, и передавать полученные в расчётах параметры в вышестоящие модели или наоборот.
\fi

\ifsynopsis
\else
Для механики твёрдого тела одна из возможных иерархий моделей проиллюстрирована на Рис. \ref{fig:ModelsHierarchy}. Согласно такому представлению модели, использующие аппарат квантовой механики \cite{QuantumModelling1, QuantumModelling2}, находятся на первой ступени иерархии, их применение ограничено масштабами сопоставимыми с ядрами атомов и простейших молекулярных соединений, состоящих из небольшого числа атомов, то есть в диапазоне от нескольких ангстрем до нескольких нанометров. На второй ступени иерархии находятся модели молекулярной динамики \cite{MD1, MD2, MD3, MD4}, такие модели могут описывать поведение сложных соединений, например, больших полимерных молекул и прочих наномасштабных объектов, размеры которых не превосходят нескольких десятков нанометров. На третьей ступени располагаются статистические модели, в частности модели, в основе которых лежит метод Монте-Карло \cite{MonteCarlo1, MonteCarlo2}. В таких моделях расчёты проводят многократно, а структуру рассматриваемого объекта генерируют случайным образом по определённым правилам, после чего полученные таким способом результаты осредняют или вычисляют на их основе вероятностные характеристики материала. И на последней --- четвёртой ступени иерархии стоят континуальные модели, в частности модели механики сплошной среды \cite{MSS}. Такие модели оперируют гипотезами сплошности среды и абсолютности времени, то есть не учитывают дискретность рассматриваемого вещества.
\fi

\ifsynopsis
\else
\begin{figure}[ht]
    \centerfloat{
        \includegraphics[width=0.85\textwidth]{pics/ModelsHierarchy.pdf}
    }
    \caption{Иерархия моделей}\label{fig:ModelsHierarchy}
\end{figure}
\fi

\ifsynopsis
\else
Однако у статистических моделей и моделей молекулярной динамики есть недостаток --- анализ объектов при помощи этих моделей без численных экспериментов крайне ограничен \cite{MDExperiment}. Поэтому с середины XX века набирают популярность модели обобщённой механики сплошной среды, которые распространяют применение моделей высшего уровня на области применения моделей низшего уровня. Одна из первых таких моделей была предложена в работе братьев Eugène и François Cosserat \cite{Cosserat}, где помимо трансляционных степеней свободы также были учтены и вращательные компоненты движения, которые связаны с трансляционными рядом соотношений, из-за чего тензор напряжений перестаёт быть симметричным. Позже, спустя пол века, эта теория была связана с теорией дислокаций в работе V.~G{\"u}nther \cite{CosseratAndDislocation} и дополнена законом сохранения микроинерции в работе A.C.~Eringen \cite{Eringen2, Eringen3}, в связи с чем теорию начали называть микрополярной теорией упругости. Также к работам, связанным с исследованием микрополярной теории упругости, можно отнести работы R.D.~Mindlin \cite{Mindlin1, Mindlin2, Mindlin3}, D.B.~Bogy \cite{Bogy} и Y.C.~Hsu \cite{Hsu}. В них авторы рассматривали применение этой теории в задачах с концентраторами, возникающим в углах и отверстиях соответсвенно. В то же время теория нашла своё отражение в работах советских учёных Э.Л.~Аэро и Е.В.~Кувшинского \cite{Aero1,Aero2}, а также была рассмотрена в работах Н.Ф.~Морозова \cite{Morozov} и Г.Н.~Савина \cite{Savin}.
\fi

\ifsynopsis
\else
Дальнейшее развитие микрополярной теории упругости привело к появлению микроморфных моделей \cite{Eringen4, Micromorph1, Micromorph2}, в которые помимо вращательных компонент движения могут быть включены дополнительные переменные, связанные с деформацией материала, при этом микрополярная теория упругости является лишь частным случаем микроморфных моделей. Но стоит учесть тот факт, что использование таких моделей сопряжено с трудностью определения материальных коэффициентов.
\fi

\ifsynopsis
\else
Список рассмотренных моделей, учитывающих вращение, а также авторов, которые занимались их развитием и исследованием, далеко не исчерпывающий. Однако стоит также уделить внимание другому классу моделей обобщённой механики сплошной среды, описывающих дальнодействующие эффекты. Это градиентные модели, которые получили своё развитие в 60-х годах XX века. Эти модели оперируют высшими производными деформаций, в связи с чем они и получили такое название. Первые модели градиентной теории упругости были сформулированы в работах Toupin~R.A. \cite{Toupin} и Mindlin~R.D. \cite{Mindlin4, Mindlin5}, которые сейчас в литературе принято называть моделями Миндлина --- Тупина \cite{ToupinMindlin1, ToupinMindlin2, ToupinMindlin3}. В работе G.~Ahmadi и K.~Firoozbakhsh \cite{GradientThermoelasticity} эти модели получили обобщение на температурные деформации. Как и микроморфные, градиентные модели обладают тем же недостатком --- большое количество материальных констант, которые необходимо определить, поэтому в 90-х годах XX века в работах E.C.~Aifantis и его соавторов \cite{Aifantis1, Aifantis2} была рассмотренна упрощёная модель градиентной теории упругости, в которой напряжения связаны с деформацией и её вторым градиентом, и добавлен всего один дополнительный материальный параметр внутренней длины.
\fi

\ifsynopsis
\else
Существует ещё один класс моделей описывающих дальнодействующие эффекты --- это нелокальные модели, которые в отличие от градиентных оперируют интегральными выражениями типа свёртки. Впервые описание таких моделей было представлено в работе E.~Kr{\"o}ner \cite{Kroner}, где были рассмотрены упругие среды с дальнодействующими силами сцепления. Модели нелокальной упругости в термодинамическом контексте были рассмотрены в работах D.G.B.~Edelen, A.E.~Green и N.~Laws \cite{Edelen1, Edelen2}, позже к их работе присоединился и A.C.~Eringen \cite{Eringen5, Eringen6}. Исследование условий, обеспечивающих существование фундаментнальных решений было проведено в работе D.~Rogula \cite{Rogula1982}. Вопросы, связанные с существованием и единственностью решений начально-краевых задач теории нелокальной упругости были рассмотрены в работах S.B.~Altan \cite{Altan1, Altan2}, позже он рассмотрел этот вопрос и для задач нелокальной термоупругости \cite{Altan3, Altan4}. В конечном итоге, в начале XXI века, A.C.~Eringen представил работу \cite{Eringen1}, в которой описан единый подход к построению нелокальных теорий для упругих тел, в связи с чем в литературе нелокальные модели часто называют моделями Эрингена и их исследованию посвящено достаточно большое количество работ \cite{BondaryLayer, Tuna, Rahmani}. При этом между нелокальными и градиентными моделями существует связь, которая была рассмотрена в работах S.B.~Altan и E.C.~Aifantis \cite{Aifantis3}, J.~Gao \cite{Gao} и др.
\fi

\ifsynopsis
\else
У нелокальных моделей также есть свои недостатки. Главный из них~--- необходимость введения дополнительных условий, так как обычных граничных условий будет недостаточно. Поэтому в прикладных исследованиях используют регуляризованные модели, в которых рассматривают комбинированные среды, состоящие из локальной и нелокальной фаз. Первые примеры рассмотрения такого рода сред можно найти в работах C.~Polizzoto \cite{Polizzotto1, Polizzotto2}, где были проанализированы одномерные задачи упругости, а также разработан численные метод решения, в основе которого был использован метод конечных элементов. Развитие этих идей в рамках двумерных задач нелокальной упругости было описано в работе A.A.~Pisano \cite{Pisano1}.
\fi

\ifsynopsis
\else
Анализ моделей нелокальной телопроводности на примере решения одномерных задач был проведён в работе Г.Н.~Кувыркина и И.Ю.~Савельевой \cite{NonlocalThermal1}. В это же время Г.Н.~Кувыркином была рассмотрена нелокальная модель термовязкоупругости в работах \cite{ThermoViscoElasticity1, ThermoViscoElasticity2, ThermoViscoElasticity3}. Позже в работах И.Ю.~Савельевой были рассмотрены задачи термоудара \cite{ThermoUdar1, ThermoUdar2, ThermoUdar3}, а также вариационные постановки задачи \cite{NonlocalThermalVariation1, NonlocalThermalVariation2}. К работе исследования задач в нелокальных постановках были подключены и ученики Г.Н.~Кувыркина в том числе и автор текущей работы \cite{AMCSM2019, ZAMM, NonlocalSaintVenant, NonlocalRadiation}.
\fi

\ifsynopsis
\else
Поиск решений задач в нелокальных постановках весьма затруднителен, так как решение интегро-дифференциальных уравнений представляет достаточно сложную задачу. В этом случае необходимо использовать различные численные методы, специально адаптированные под данный класс уравнений. В этом направлении есть уже достаточно большое количество работ, предлагающие использовать различные методы решения. Наиболее общим и популярным является метод конечных элементов (FEM), который применительно к данному классу уравнений иногда ещё называют методом нелокальных конечных элементов (NL-FEM) \cite{Polizzotto2, Pisano1}. Однако его использование сопряжёно с большой вычислительной сложностью. Поэтому некоторые исследователи используют возможности интегральных преобразований откуда был получен метод на основе быстрого преобразования Гаусса (FEMFGT) \cite{FastGaussTransform}. Но использование этого метода сопряжено с рядом трудностей, так как для его применения необходима достаточно подробная сетка, чтобы избежать возможных осциляций решения, но при этом всё равно сохраняется проблема контроля точности решения. Помимо сеточных методов большой популярностью пользуются и бессеточные подходы на основе радиальных базисных функций \cite{RadialBasis}, безэлементный метод Галёркина (EFG) и метод конечных точек (FPM) \cite{MeshFree}. Также были предложены подходы с использованием пограничного слоя \cite{BondaryLayer} и на основе полиномов Чебышёва \cite{ChebPolynom}.
\fi

\ifsynopsis
\else
В рамках текущей работы было принято решение использовать метод нелокальных конечных элементов, так как данный метод достаточно хорошо изучен и его легко использовать на областях со сложной геометрией, а большое количество редакторов и генераторов сеток упрощает процесс моделирования. Также в работе проведена большая работа по ускорению данного метода, но чтобы в полной мере реализовать весь потенциал предложенных алгоритмов пришлось реализовать свой собственный програмный комплекс NonLocFEM \cite{NonLocFEM} на языке программирования C++. Такое решение связано с тем, что многие современные коммерческие программные комплексы, например Ansys \cite{Ansys}, Abaqus \cite{Abaqus}, TFlex \cite{TFlex} и др., имеют закрытый программный код. С другой стороны существуют открытые программные комплексы, например, Deal.II \cite{DealII}, FEniCS \cite{FEniCS}, FreeFEM \cite{FreeFEM} и многие другие, но при использовании такого рода программных комплексов могут возникнуть ситуации, когда нет возможности эффективно реализовать тот или иной алгоритм в силу принципов заложенных в их основу, а также возможное возникновение конфликтов интересов, так как многие программные комплексы созданы при поддержке зарубежных инвесторов, которые в любой момент могут ограничить к ним доступ.
\fi

%\ifsynopsis
%Этот абзац появляется только в~автореферате.
%\else
%Этот абзац появляется только в~диссертации.
%\fi

% {\progress}
% Этот раздел должен быть отдельным структурным элементом по
% ГОСТ, но он, как правило, включается в описание актуальности
% темы. Нужен он отдельным структурынм элемементом или нет ---
% смотрите другие диссертации вашего совета, скорее всего не нужен.

{\aim} исследования является изучение особенностей разработанных двумерных моделей нелокальной теплопроводности и термоупругости, а также сравнительный анализ решений в случае классических и нелокальных моделей механики сплошной среды.

Для достижения поставленной цели потребовалось решить {\tasks}.
\begin{enumerate}[beginpenalty=10000] % https://tex.stackexchange.com/a/476052/104425
  \item Разработать определяющие соотношения двумерных моделей теплопроводности и термоупругости нелокальной среды в интегро-дифференциальной форме.
  \item Разработать алгоритмы численного решения на основе метода конечных элементов с последующей оптимизацией для более эффективного использования на многопроцессорных вычислительных машинах.
  \item Разработать экономичные способы предобуславливания получаемых при аппроксимации систем линейных алгебраических уравнений (СЛАУ) с целью ускорения сходимости итерационных методов решения.
  \item Реализовать полученные алгоритмы в виде программного комплекса.
  \item Исследовать особенности нелокальных моделей, сопоставить полученные результаты в задачах с известными решениями в классической постановке, определить закономерности.
\end{enumerate}


{\novelty}
\begin{enumerate}[beginpenalty=10000] % https://tex.stackexchange.com/a/476052/104425
  \item Предложены новые эффективные численные алгоритмы для задач нелокальной теплопроводности и нелокальной термоупругости на основе метода конечных элементов, которые обладают хорошей масштабируемостью и предназначены для вычислений на многопроцессорных вычислительных машинах с общей и распределённой памятью.
  \item Разработан собственный программный комплекс NonLocFEM, в котором реализованы все представленные в работе алгоритмы и методы для моделирования поведения структурно-чувствительных материалов.
  \item Получены новые результаты в задачах с известными для классической постановки решениями, установлены закономерности, свидетельствующие о снижении роли концентраторов в распределениях полей напряжений и плотности теплового потока.
  %\item Проведено качественное сравнение между результатами полученными с использованием классической и нелокальной теориями, которые свидетельствуют о снижении роли концентраторов в распределениях полей напряжений и плотности теплового потока.
  \item Исследованы границы спектров собственных чисел матриц и установлены связи между спектрами матриц, ассемблированных в классической и нелокальной постановках.
\end{enumerate}

{\influence}
моделей, рассмотренных в диссертации, состоит в возможности описания поведения термомеханических состояний структурно-чувствительных материалов. Параметры модели очевидным образом влияют на решения, что дает возможность точно настраивать модель для применения на практике. Разработанный программный комплекс, в котором реализованы численные алгоритмы исследования разработанных моделей, позволит проводить расчёты на произвольных областях со всеми рассматриваемыми в моделе параметрами, а благодаря открытому исходному коду и модульной структуре существует возможность редактировать существующие постановки и добавлять новые типы расчётов при модификации математической модели.

{\methods}
В диссертации использованы как классические принципы механики деформируемого твёрдого тела, так и новые, относящиеся к нелокальным теориям теплопроводности и термоупругости, а также численные методы, в основе которых лежит метод конечных элементов.

{\defpositions}
\begin{enumerate}[beginpenalty=10000] % https://tex.stackexchange.com/a/476052/104425
 	\item Модели нелокальной теплопроводности и термоупругости, позволяющие описать процессы передачи теплоты и напряжённо-деформированного состояния в структурно-чувствительных материалах.
	\item Новые численные алгоритмы решения на основе метода конечных элементов, адапатированные под многопроцессорные вычислительные системы.
	\item Собственный программный комплекс NonLocFEM, в рамках которого реализованы все рассматриваемые в работе методы решений.
\end{enumerate}


{\reliability} результатов гарантирована строгостью и полнотой использования возможностей математического аппарата, сравнением результатов многочисленных проведеннных расчетов с известными аналитическими решениями и данными полученными ранее другими авторами.
%{\reliability} полученных результатов гарантирует строгость используемого математического аппарата, сравнение расчётов с известными аналитическими решениями и данными полученными ранее другими авторами.

\ifsynopsis
{\probation} проводилась в обсуждениях на следующих конференциях:
Международная научно-техническая конференция <<Актуальные проблемы прикладной математики, информатикии и механики>> (Воронеж, 2019, 2021); Международная конференция <<International Conference of Numerical Analysis and Applied Mathematics>> (Родос, Греция, 2021); Международная научная конференция <<Фундаментальные и Прикладные Задачи Механики>> (Москва, 2021); Всероссийская конференция по численным методам решения задач теории упругости и пластичности (Красноярск, 2023); Международная конференция <<Математическое моделирование, численные методы и инженерное программное обеспечение>> (Москва, 2023).
\else
{\probation} проводилась в обсуждениях на следующих конференциях:
\begin{enumerate}
	\item Международная научно-техническая конференция <<Актуальные проблемы прикладной математики, информатикии и механики>> (Воронеж, 2019, 2021);
	\item Международная конференция <<International Conference of Numerical Analysis and Applied Mathematics>> (Родос, Греция, 2021);
	\item Международная научная конференция <<Фундаментальные и Прикладные Задачи Механики>> (Москва, 2021);
	\item Всероссийская конференция по численным методам решения задач теории упругости и пластичности (Красноярск, 2023);
	\item Математическое моделирование, численные методы и инженерное программное обеспечение (Москва, 2023).
\end{enumerate}
\fi

\ifsynopsis
Диссертация является состaвной чaстью фундаментальных исследований, выполненных в рамках грантов: 0705-2020-0047 <<Теория дифференциальных уравнений, краевые задачи, связанные задачи анализа и теории приближений и некоторые их приложения>>; FSFN-2023-0012 «Разработка математических моделей и методов проектирования изделий ракетно-космической техники из перспективных конструкционных и функциональных материалов»; FSFN-2024-0004 <<Разработка математических моделей и методов проектирования изделий ракетно-космической техники из перспективных конструкционных и функциональных материалов>>.
\else
Диссертация является состaвной чaстью фундаментальных исследований, выполненных в рамках грантов.
\begin{enumerate}
	\item 0705-2020-0047 <<Теория дифференциальных уравнений, краевые задачи, связанные задачи анализа и теории приближений и некоторые их приложения>>.
	\item FSFN-2023-0012 <<Разработка математических моделей и методов проектирования изделий ракетно-космической техники из перспективных конструкционных и функциональных материалов>>.
	\item FSFN-2024-0004 <<Разработка математических моделей и методов проектирования изделий ракетно-космической техники из перспективных конструкционных и функциональных материалов>>.
\end{enumerate}
\fi

\ifnumequal{\value{bibliosel}}{0}
{%%% Встроенная реализация с загрузкой файла через движок bibtex8. (При желании, внутри можно использовать обычные ссылки, наподобие `\cite{vakbib1,vakbib2}`).
    {\publications} Основные результаты по теме диссертации изложены
    в~XX~печатных изданиях,
    X из которых изданы в журналах, рекомендованных ВАК,
    X "--- в тезисах докладов.
}%
{%%% Реализация пакетом biblatex через движок biber
    \begin{refsection}[bl-author, bl-registered]
        % Это refsection=1.
        % Процитированные здесь работы:
        %  * подсчитываются, для автоматического составления фразы "Основные результаты ..."
        %  * попадают в авторскую библиографию, при usefootcite==0 и стиле `\insertbiblioauthor` или `\insertbiblioauthorgrouped`
        %  * нумеруются там в зависимости от порядка команд `\printbibliography` в этом разделе.
        %  * при использовании `\insertbiblioauthorgrouped`, порядок команд `\printbibliography` в нём должен быть тем же (см. biblio/biblatex.tex)
        %
        % Невидимый библиографический список для подсчёта количества публикаций:
        \printbibliography[heading=nobibheading, section=1, env=countauthorvak,          keyword=biblioauthorvak]%
        \printbibliography[heading=nobibheading, section=1, env=countauthorwos,          keyword=biblioauthorwos]%
        \printbibliography[heading=nobibheading, section=1, env=countauthorscopus,       keyword=biblioauthorscopus]%
        \printbibliography[heading=nobibheading, section=1, env=countauthorconf,         keyword=biblioauthorconf]%
        \printbibliography[heading=nobibheading, section=1, env=countauthorother,        keyword=biblioauthorother]%
        \printbibliography[heading=nobibheading, section=1, env=countregistered,         keyword=biblioregistered]%
        \printbibliography[heading=nobibheading, section=1, env=countauthorpatent,       keyword=biblioauthorpatent]%
        \printbibliography[heading=nobibheading, section=1, env=countauthorprogram,      keyword=biblioauthorprogram]%
        \printbibliography[heading=nobibheading, section=1, env=countauthor,             keyword=biblioauthor]%
        \printbibliography[heading=nobibheading, section=1, env=countauthorvakscopuswos, filter=vakscopuswos]%
        \printbibliography[heading=nobibheading, section=1, env=countauthorscopuswos,    filter=scopuswos]%
        %
        \nocite{*}%
        %
        {\publications} Основные результаты по теме диссертации изложены в~\arabic{citeauthor}~печатных изданиях,
        \arabic{citeauthorvak} из которых изданы в журналах, рекомендованных ВАК \sloppy%
        \ifnum \value{citeauthorscopuswos}>0%
            , \arabic{citeauthorscopuswos} "--- в~периодических научных журналах, индексируемых Web of~Science и Scopus\sloppy%
        \fi%
        \ifnum \value{citeauthorconf}>0%
            , \arabic{citeauthorconf} "--- в~тезисах докладов.
        \else%
            .
        \fi%
        \ifnum \value{citeregistered}=1%
            \ifnum \value{citeauthorpatent}=1%
                Зарегистрирован \arabic{citeauthorpatent} патент.
            \fi%
            \ifnum \value{citeauthorprogram}=1%
                Зарегистрирована \arabic{citeauthorprogram} программа для ЭВМ.
            \fi%
        \fi%
        \ifnum \value{citeregistered}>1%
            Зарегистрированы\ %
            \ifnum \value{citeauthorpatent}>0%
            \formbytotal{citeauthorpatent}{патент}{}{а}{}\sloppy%
            \ifnum \value{citeauthorprogram}=0 . \else \ и~\fi%
            \fi%
            \ifnum \value{citeauthorprogram}>0%
            \formbytotal{citeauthorprogram}{программ}{а}{ы}{} для ЭВМ.
            \fi%
        \fi%
        % К публикациям, в которых излагаются основные научные результаты диссертации на соискание учёной
        % степени, в рецензируемых изданиях приравниваются патенты на изобретения, патенты (свидетельства) на
        % полезную модель, патенты на промышленный образец, патенты на селекционные достижения, свидетельства
        % на программу для электронных вычислительных машин, базу данных, топологию интегральных микросхем,
        % зарегистрированные в установленном порядке.(в ред. Постановления Правительства РФ от 21.04.2016 N 335)
    \end{refsection}%
    \begin{refsection}[bl-author, bl-registered]
        % Это refsection=2.
        % Процитированные здесь работы:
        %  * попадают в авторскую библиографию, при usefootcite==0 и стиле `\insertbiblioauthorimportant`.
        %  * ни на что не влияют в противном случае
%        \nocite{vakbib2}%vak
%        \nocite{patbib1}%patent
%        \nocite{progbib1}%program
%        \nocite{bib1}%other
%        \nocite{confbib1}%conf
    \end{refsection}%
        %
        % Всё, что вне этих двух refsection, это refsection=0,
        %  * для диссертации - это нормальные ссылки, попадающие в обычную библиографию
        %  * для автореферата:
        %     * при usefootcite==0, ссылка корректно сработает только для источника из `external.bib`. Для своих работ --- напечатает "[0]" (и даже Warning не вылезет).
        %     * при usefootcite==1, ссылка сработает нормально. В авторской библиографии будут только процитированные в refsection=0 работы.
}

{\contribution}
Все исследования, представленные в диссертационной работе, а также разработка программного комплекса выполнены лично соискателем в процессе научной деятельности. Из совместных публикаций в диссертацию включен лишь тот материал, который принадлежит соискателю, заимствованный материал обозначен в работе ссылками.

%При использовании пакета \verb!biblatex! будут подсчитаны все работы, добавленные
%в файл \verb!biblio/author.bib!. Для правильного подсчёта работ в~различных
%системах цитирования требуется использовать поля:
%\begin{itemize}
%        \item \texttt{authorvak} если публикация индексирована ВАК,
%        \item \texttt{authorscopus} если публикация индексирована Scopus,
%        \item \texttt{authorwos} если публикация индексирована Web of Science,
%        \item \texttt{authorconf} для докладов конференций,
%        \item \texttt{authorpatent} для патентов,
%        \item \texttt{authorprogram} для зарегистрированных программ для ЭВМ,
%        \item \texttt{authorother} для других публикаций.
%\end{itemize}
%Для подсчёта используются счётчики:
%\begin{itemize}
%        \item \texttt{citeauthorvak} для работ, индексируемых ВАК,
%        \item \texttt{citeauthorscopus} для работ, индексируемых Scopus,
%        \item \texttt{citeauthorwos} для работ, индексируемых Web of Science,
%        \item \texttt{citeauthorvakscopuswos} для работ, индексируемых одной из трёх баз,
%        \item \texttt{citeauthorscopuswos} для работ, индексируемых Scopus или Web of~Science,
%        \item \texttt{citeauthorconf} для докладов на конференциях,
%        \item \texttt{citeauthorother} для остальных работ,
%        \item \texttt{citeauthorpatent} для патентов,
%        \item \texttt{citeauthorprogram} для зарегистрированных программ для ЭВМ,
%        \item \texttt{citeauthor} для суммарного количества работ.
%\end{itemize}
% Счётчик \texttt{citeexternal} используется для подсчёта процитированных публикаций;
% \texttt{citeregistered} "--- для подсчёта суммарного количества патентов и программ для ЭВМ.

%Для добавления в список публикаций автора работ, которые не были процитированы в
%автореферате, требуется их~перечислить с использованием команды \verb!\nocite! в
%\verb!Synopsis/content.tex!.
